%% Generated by Sphinx.
\def\sphinxdocclass{report}
\documentclass[letterpaper,10pt,english]{sphinxmanual}
\ifdefined\pdfpxdimen
   \let\sphinxpxdimen\pdfpxdimen\else\newdimen\sphinxpxdimen
\fi \sphinxpxdimen=.75bp\relax
\ifdefined\pdfimageresolution
    \pdfimageresolution= \numexpr \dimexpr1in\relax/\sphinxpxdimen\relax
\fi
%% let collapsible pdf bookmarks panel have high depth per default
\PassOptionsToPackage{bookmarksdepth=5}{hyperref}

\PassOptionsToPackage{warn}{textcomp}
\usepackage[utf8]{inputenc}
\ifdefined\DeclareUnicodeCharacter
% support both utf8 and utf8x syntaxes
  \ifdefined\DeclareUnicodeCharacterAsOptional
    \def\sphinxDUC#1{\DeclareUnicodeCharacter{"#1}}
  \else
    \let\sphinxDUC\DeclareUnicodeCharacter
  \fi
  \sphinxDUC{00A0}{\nobreakspace}
  \sphinxDUC{2500}{\sphinxunichar{2500}}
  \sphinxDUC{2502}{\sphinxunichar{2502}}
  \sphinxDUC{2514}{\sphinxunichar{2514}}
  \sphinxDUC{251C}{\sphinxunichar{251C}}
  \sphinxDUC{2572}{\textbackslash}
\fi
\usepackage{cmap}
\usepackage[T1]{fontenc}
\usepackage{amsmath,amssymb,amstext}
\usepackage{babel}



\usepackage{tgtermes}
\usepackage{tgheros}
\renewcommand{\ttdefault}{txtt}



\usepackage[Bjarne]{fncychap}
\usepackage{sphinx}

\fvset{fontsize=auto}
\usepackage{geometry}


% Include hyperref last.
\usepackage{hyperref}
% Fix anchor placement for figures with captions.
\usepackage{hypcap}% it must be loaded after hyperref.
% Set up styles of URL: it should be placed after hyperref.
\urlstyle{same}

\addto\captionsenglish{\renewcommand{\contentsname}{Contents:}}

\usepackage{sphinxmessages}
\setcounter{tocdepth}{1}



\title{Digital plots of Modulus and Strength ratio for Rocks}
\date{Dec 19, 2022}
\release{0.0}
\author{Grasselli\textquotesingle{}s Geomechanics Group \sphinxhyphen{} University of Toronto}
\newcommand{\sphinxlogo}{\vbox{}}
\renewcommand{\releasename}{Release}
\makeindex
\begin{document}

\ifdefined\shorthandoff
  \ifnum\catcode`\=\string=\active\shorthandoff{=}\fi
  \ifnum\catcode`\"=\active\shorthandoff{"}\fi
\fi

\pagestyle{empty}
\sphinxmaketitle
\pagestyle{plain}
\sphinxtableofcontents
\pagestyle{normal}
\phantomsection\label{\detokenize{index::doc}}


\sphinxstepscope


\chapter{Introduction}
\label{\detokenize{postprocessing_intro:introduction}}\label{\detokenize{postprocessing_intro::doc}}
\sphinxAtStartPar
Makes it easier to classify the Deere\sphinxhyphen{}Miller \sphinxhyphen{} Modulus Ratio {[}MR{]} and Tatone et al. \sphinxhyphen{} Strength Ratio {[}SR{]}.

\sphinxAtStartPar
For any suggestions, bugs or if you wish to contribute to the project =\textgreater{} \sphinxhref{https://github.com/alicarlos/digital\_modulus\_strength\_ratio}{REPO}


\section{Deere\sphinxhyphen{}Miller \sphinxhyphen{} Modulus Ratio (MR)}
\label{\detokenize{postprocessing_intro:deere-miller-modulus-ratio-mr}}
\sphinxAtStartPar
\sphinxstyleemphasis{Cite: Deere DU, Miller RP. Engineering Classification and Index Properties for Intact Rocks. Fort Belvoir, VA: Defense Technical Information Center; 1966.}

\sphinxAtStartPar
Loads the digitized Deere\_Miller clusters and plots them based on the Major Rock Type \sphinxstyleemphasis{(i.e., Igneous / Metamorphic / Sedimentary)}.
\begin{itemize}
\item {} 
\sphinxAtStartPar
Plot all Major Rock Type in one graph.

\item {} 
\sphinxAtStartPar
Plots them individually.

\end{itemize}

\sphinxAtStartPar
\sphinxstylestrong{Deere\sphinxhyphen{}Miller {[}Modulus Ratio{]} Example}
\begin{enumerate}
\sphinxsetlistlabels{\arabic}{enumi}{enumii}{}{.}%
\item {} 
\sphinxAtStartPar
Plot the Modulus Ratio of just the Sedimentary clusters with the ISRM 1979 category classification.

\end{enumerate}

\begin{sphinxVerbatim}[commandchars=\\\{\}]
\PYG{k+kn}{import} \PYG{n+nn}{pyrockmodulus}
\PYG{k+kn}{import} \PYG{n+nn}{matplotlib}\PYG{n+nn}{.}\PYG{n+nn}{pyplot} \PYG{k}{as} \PYG{n+nn}{plt}

\PYG{n}{xx} \PYG{o}{=} \PYG{n}{pyrockmodulus}\PYG{o}{.}\PYG{n}{modulus\PYGZus{}ratio}\PYG{p}{(}\PYG{p}{)}
\PYG{n}{xx}\PYG{o}{.}\PYG{n}{initial\PYGZus{}processing}\PYG{p}{(}\PYG{n}{plot\PYGZus{}all\PYGZus{}clusters}\PYG{o}{=}\PYG{k+kc}{False}\PYG{p}{,} \PYG{n}{rock\PYGZus{}type\PYGZus{}to\PYGZus{}plot}\PYG{o}{=}\PYG{l+s+s1}{\PYGZsq{}}\PYG{l+s+s1}{Sedimentary}\PYG{l+s+s1}{\PYGZsq{}}\PYG{p}{,} \PYG{n}{ucs\PYGZus{}class\PYGZus{}type}\PYG{o}{=}\PYG{l+s+s2}{\PYGZdq{}}\PYG{l+s+s2}{ISRMCAT}\PYG{l+s+se}{\PYGZbs{}n}\PYG{l+s+s2}{1979}\PYG{l+s+s2}{\PYGZdq{}}\PYG{p}{)}
\PYG{n}{plt}\PYG{o}{.}\PYG{n}{ylabel}\PYG{p}{(}\PYG{l+s+s2}{\PYGZdq{}}\PYG{l+s+s2}{E (GPa)}\PYG{l+s+s2}{\PYGZdq{}}\PYG{p}{)}
\PYG{n}{plt}\PYG{o}{.}\PYG{n}{xlabel}\PYG{p}{(}\PYG{l+s+s2}{\PYGZdq{}}\PYG{l+s+s2}{UCS (MPa)}\PYG{l+s+s2}{\PYGZdq{}}\PYG{p}{)}
\PYG{n}{plt}\PYG{o}{.}\PYG{n}{show}\PYG{p}{(}\PYG{p}{)}
\end{sphinxVerbatim}

\sphinxAtStartPar
\sphinxincludegraphics{{example01}.png}
\begin{enumerate}
\sphinxsetlistlabels{\arabic}{enumi}{enumii}{}{.}%
\setcounter{enumi}{1}
\item {} 
\sphinxAtStartPar
Plot the Modulus Ratio with all the categories without the classification. Legend enabled.

\end{enumerate}

\begin{sphinxVerbatim}[commandchars=\\\{\}]
\PYG{k+kn}{import} \PYG{n+nn}{pyrockmodulus}
\PYG{k+kn}{import} \PYG{n+nn}{matplotlib}\PYG{n+nn}{.}\PYG{n+nn}{pyplot} \PYG{k}{as} \PYG{n+nn}{plt}

\PYG{n}{xx} \PYG{o}{=} \PYG{n}{pyrockmodulus}\PYG{o}{.}\PYG{n}{modulus\PYGZus{}ratio}\PYG{p}{(}\PYG{p}{)}
\PYG{n}{xx}\PYG{o}{.}\PYG{n}{initial\PYGZus{}processing}\PYG{p}{(}\PYG{n}{plot\PYGZus{}all\PYGZus{}clusters}\PYG{o}{=}\PYG{k+kc}{True}\PYG{p}{)}
\PYG{n}{plt}\PYG{o}{.}\PYG{n}{ylabel}\PYG{p}{(}\PYG{l+s+s2}{\PYGZdq{}}\PYG{l+s+s2}{E (GPa)}\PYG{l+s+s2}{\PYGZdq{}}\PYG{p}{)}
\PYG{n}{plt}\PYG{o}{.}\PYG{n}{xlabel}\PYG{p}{(}\PYG{l+s+s2}{\PYGZdq{}}\PYG{l+s+s2}{UCS (MPa)}\PYG{l+s+s2}{\PYGZdq{}}\PYG{p}{)}
\PYG{n}{plt}\PYG{o}{.}\PYG{n}{legend}\PYG{p}{(}\PYG{p}{)}
\PYG{n}{plt}\PYG{o}{.}\PYG{n}{show}\PYG{p}{(}\PYG{p}{)}
\end{sphinxVerbatim}

\sphinxAtStartPar
\sphinxincludegraphics{{example02}.png}
\begin{enumerate}
\sphinxsetlistlabels{\arabic}{enumi}{enumii}{}{.}%
\setcounter{enumi}{2}
\item {} 
\sphinxAtStartPar
Plot the Modulus Ratio of just the Sedimentary clusters overlaid with data from tests.

\end{enumerate}

\begin{sphinxVerbatim}[commandchars=\\\{\}]
\PYG{k+kn}{import} \PYG{n+nn}{pyrockmodulus}
\PYG{k+kn}{import} \PYG{n+nn}{matplotlib}\PYG{n+nn}{.}\PYG{n+nn}{pyplot} \PYG{k}{as} \PYG{n+nn}{plt}
\PYG{c+c1}{\PYGZsh{} Data Set}
\PYG{n}{ucs\PYGZus{}data} \PYG{o}{=} \PYG{p}{[}\PYG{l+m+mf}{75.33}\PYG{p}{,} \PYG{l+m+mf}{99.03}\PYG{p}{,} \PYG{l+m+mf}{111.69}\PYG{p}{,} \PYG{l+m+mf}{30.17}\PYG{p}{,} \PYG{l+m+mf}{73.76}\PYG{p}{,} \PYG{l+m+mf}{41.69}\PYG{p}{,} \PYG{l+m+mf}{42.09}\PYG{p}{,} \PYG{l+m+mf}{60.99}\PYG{p}{,} \PYG{l+m+mf}{39.65}\PYG{p}{,} \PYG{l+m+mf}{94.52}\PYG{p}{,} \PYG{l+m+mf}{104.6}\PYG{p}{,} \PYG{l+m+mf}{102.03}\PYG{p}{]}
\PYG{n}{E\PYGZus{}data} \PYG{o}{=} \PYG{p}{[}\PYG{l+m+mf}{18.31}\PYG{p}{,} \PYG{l+m+mf}{21.85}\PYG{p}{,} \PYG{l+m+mf}{20.51}\PYG{p}{,} \PYG{l+m+mf}{8.62}\PYG{p}{,} \PYG{l+m+mf}{25.72}\PYG{p}{,} \PYG{l+m+mf}{18.68}\PYG{p}{,} \PYG{l+m+mf}{9.2}\PYG{p}{,} \PYG{l+m+mf}{14.67}\PYG{p}{,} \PYG{l+m+mf}{7.38}\PYG{p}{,} \PYG{l+m+mf}{8.48}\PYG{p}{,} \PYG{l+m+mf}{8.7}\PYG{p}{,} \PYG{l+m+mf}{8.82}\PYG{p}{]}
\PYG{n}{xx} \PYG{o}{=} \PYG{n}{pyrockmodulus}\PYG{o}{.}\PYG{n}{modulus\PYGZus{}ratio}\PYG{p}{(}\PYG{p}{)}
\PYG{n}{plotting\PYGZus{}axis} \PYG{o}{=} \PYG{n}{xx}\PYG{o}{.}\PYG{n}{initial\PYGZus{}processing}\PYG{p}{(}\PYG{n}{rock\PYGZus{}type\PYGZus{}to\PYGZus{}plot}\PYG{o}{=}\PYG{l+s+s1}{\PYGZsq{}}\PYG{l+s+s1}{Sedimentary}\PYG{l+s+s1}{\PYGZsq{}}\PYG{p}{)}
\PYG{c+c1}{\PYGZsh{} Plot the data on the Deere\PYGZhy{}Miller axis}
\PYG{n}{plotting\PYGZus{}axis}\PYG{o}{.}\PYG{n}{scatter}\PYG{p}{(}\PYG{n}{ucs\PYGZus{}data}\PYG{p}{,} \PYG{n}{E\PYGZus{}data}\PYG{p}{,} \PYG{n}{label}\PYG{o}{=}\PYG{l+s+s1}{\PYGZsq{}}\PYG{l+s+s1}{Test Results}\PYG{l+s+s1}{\PYGZsq{}}\PYG{p}{,} \PYG{n}{marker}\PYG{o}{=}\PYG{l+s+s1}{\PYGZsq{}}\PYG{l+s+s1}{.}\PYG{l+s+s1}{\PYGZsq{}}\PYG{p}{)}
\PYG{n}{plt}\PYG{o}{.}\PYG{n}{ylabel}\PYG{p}{(}\PYG{l+s+s2}{\PYGZdq{}}\PYG{l+s+s2}{E (GPa)}\PYG{l+s+s2}{\PYGZdq{}}\PYG{p}{)}
\PYG{n}{plt}\PYG{o}{.}\PYG{n}{xlabel}\PYG{p}{(}\PYG{l+s+s2}{\PYGZdq{}}\PYG{l+s+s2}{UCS (MPa)}\PYG{l+s+s2}{\PYGZdq{}}\PYG{p}{)}
\PYG{n}{plt}\PYG{o}{.}\PYG{n}{legend}\PYG{p}{(}\PYG{p}{)}
\PYG{n}{plt}\PYG{o}{.}\PYG{n}{show}\PYG{p}{(}\PYG{p}{)}
\end{sphinxVerbatim}

\sphinxAtStartPar
\sphinxincludegraphics{{example_withdata}.png}


\section{Tatone et al. \sphinxhyphen{} Strength Ratio (SR)}
\label{\detokenize{postprocessing_intro:tatone-et-al-strength-ratio-sr}}
\sphinxAtStartPar
\sphinxstyleemphasis{Tatone, B.S.A., Abdelaziz, A. \& Grasselli, G. Novel Mechanical Classification Method of Rock Based on the Uniaxial Compressive Strength and Brazilian Disc Strength. Rock Mech Rock Eng 55, 2503\textendash{}2507 (2022). https://doi.org/10.1007/s00603\sphinxhyphen{}021\sphinxhyphen{}02759\sphinxhyphen{}7}

\sphinxAtStartPar
Loads the constructed Tatone et al. UCS:BDS clusters and plots them based on the Major Rock Type \sphinxstyleemphasis{(i.e., Igneous / Metamorphic / Sedimentary)}.
\begin{itemize}
\item {} 
\sphinxAtStartPar
Plot all Major Rock Type in one graph.

\item {} 
\sphinxAtStartPar
Plots them individually.

\end{itemize}

\sphinxAtStartPar
The functionality is similar to that of the modulus ratio.

\begin{sphinxVerbatim}[commandchars=\\\{\}]
\PYG{k+kn}{import} \PYG{n+nn}{pyrockmodulus}
\PYG{k+kn}{import} \PYG{n+nn}{matplotlib}\PYG{n+nn}{.}\PYG{n+nn}{pyplot} \PYG{k}{as} \PYG{n+nn}{plt}

\PYG{n}{xx} \PYG{o}{=} \PYG{n}{pyrockmodulus}\PYG{o}{.}\PYG{n}{strength\PYGZus{}ratio}\PYG{p}{(}\PYG{p}{)}
\PYG{n}{xx}\PYG{o}{.}\PYG{n}{initial\PYGZus{}processing}\PYG{p}{(}\PYG{n}{plot\PYGZus{}all\PYGZus{}clusters}\PYG{o}{=}\PYG{k+kc}{False}\PYG{p}{,} \PYG{n}{rock\PYGZus{}type\PYGZus{}to\PYGZus{}plot}\PYG{o}{=}\PYG{l+s+s1}{\PYGZsq{}}\PYG{l+s+s1}{Sedimentary}\PYG{l+s+s1}{\PYGZsq{}}\PYG{p}{)}
\PYG{n}{plt}\PYG{o}{.}\PYG{n}{ylabel}\PYG{p}{(}\PYG{l+s+s2}{\PYGZdq{}}\PYG{l+s+s2}{BDS (MPa)}\PYG{l+s+s2}{\PYGZdq{}}\PYG{p}{)}
\PYG{n}{plt}\PYG{o}{.}\PYG{n}{xlabel}\PYG{p}{(}\PYG{l+s+s2}{\PYGZdq{}}\PYG{l+s+s2}{UCS (MPa)}\PYG{l+s+s2}{\PYGZdq{}}\PYG{p}{)}
\PYG{n}{plt}\PYG{o}{.}\PYG{n}{show}\PYG{p}{(}\PYG{p}{)}
\end{sphinxVerbatim}

\sphinxAtStartPar
\sphinxincludegraphics{{example06}.png}


\section{Poisson’s Ratio and Density Plots}
\label{\detokenize{postprocessing_intro:poisson-s-ratio-and-density-plots}}
\sphinxAtStartPar
Plot the most common ranges of density and poisson’s ratio for rock. This data can then be overlaid with data from a specific source to show comparison.

\begin{sphinxVerbatim}[commandchars=\\\{\}]
\PYG{k+kn}{import} \PYG{n+nn}{matplotlib}\PYG{n+nn}{.}\PYG{n+nn}{pyplot} \PYG{k}{as} \PYG{n+nn}{plt}
\PYG{k+kn}{import} \PYG{n+nn}{pyrockmodulus}
\PYG{n}{xx} \PYG{o}{=} \PYG{n}{pyrockmodulus}\PYG{o}{.}\PYG{n}{poisson\PYGZus{}density}\PYG{p}{(}\PYG{p}{)}
\PYG{n}{df\PYGZus{}data} \PYG{o}{=} \PYG{n}{xx}\PYG{o}{.}\PYG{n}{initial\PYGZus{}processing}\PYG{p}{(}\PYG{p}{)}
\PYG{n}{ax1} \PYG{o}{=} \PYG{n}{xx}\PYG{o}{.}\PYG{n}{plot\PYGZus{}span\PYGZus{}chart}\PYG{p}{(}\PYG{n}{df\PYGZus{}data}\PYG{p}{,} \PYG{p}{[}\PYG{l+s+s1}{\PYGZsq{}}\PYG{l+s+s1}{Min\PYGZus{}D}\PYG{l+s+s1}{\PYGZsq{}}\PYG{p}{,} \PYG{l+s+s1}{\PYGZsq{}}\PYG{l+s+s1}{Max\PYGZus{}D}\PYG{l+s+s1}{\PYGZsq{}}\PYG{p}{]}\PYG{p}{,} \PYG{l+s+s1}{\PYGZsq{}}\PYG{l+s+s1}{Density}\PYG{l+s+s1}{\PYGZsq{}}\PYG{p}{,} \PYG{l+s+sa}{r}\PYG{l+s+s1}{\PYGZsq{}}\PYG{l+s+s1}{\PYGZdl{}}\PYG{l+s+s1}{\PYGZbs{}}\PYG{l+s+s1}{rho\PYGZdl{} g/cm\PYGZdl{}\PYGZca{}}\PYG{l+s+si}{\PYGZob{}3\PYGZcb{}}\PYG{l+s+s1}{\PYGZdl{}}\PYG{l+s+s1}{\PYGZsq{}}\PYG{p}{)}
\PYG{n}{ax1}\PYG{o}{.}\PYG{n}{axvline}\PYG{p}{(}\PYG{l+m+mf}{2.0}\PYG{p}{,} \PYG{n}{lw}\PYG{o}{=}\PYG{l+m+mi}{1}\PYG{p}{,} \PYG{n}{ls}\PYG{o}{=}\PYG{l+s+s1}{\PYGZsq{}}\PYG{l+s+s1}{\PYGZhy{}\PYGZhy{}}\PYG{l+s+s1}{\PYGZsq{}}\PYG{p}{)}
\PYG{n}{plt}\PYG{o}{.}\PYG{n}{show}\PYG{p}{(}\PYG{p}{)}
\end{sphinxVerbatim}

\sphinxAtStartPar
\sphinxincludegraphics{{example_PR_DEN}.png}


\section{UCS Classification Systems}
\label{\detokenize{postprocessing_intro:ucs-classification-systems}}
\sphinxAtStartPar
This file holds the dictionaries for the various UCS classification systems available. References for those systems are within the file. All values \sphinxstylestrong{must} be in \sphinxstylestrong{MPa}.
Available classification systems \sphinxstyleemphasis{‘ISRM\textbackslash{}n1977’, ‘ISRMCAT\textbackslash{}n1979’, ‘Bieniawski\textbackslash{}n1974’, ‘Jennings\textbackslash{}n1973’, ‘Broch \& Franklin\textbackslash{}n1972’, ‘Geological Society\textbackslash{}n1970’, ‘Deere \& Miller\textbackslash{}n1966’, ‘Coates\textbackslash{}n1964’, ‘Coates \& Parsons\textbackslash{}n1966’, ‘ISO 14689\textbackslash{}n2017’, ‘Anon\textbackslash{}n1977’, ‘Anon\textbackslash{}n1979’, ‘Ramamurthy\textbackslash{}n2004’}

\sphinxAtStartPar
\sphinxstylestrong{UCS Classification System Examples}
\begin{enumerate}
\sphinxsetlistlabels{\arabic}{enumi}{enumii}{}{.}%
\item {} 
\sphinxAtStartPar
Display the limits and the classification system default in the script.

\end{enumerate}

\begin{sphinxVerbatim}[commandchars=\\\{\}]
\PYG{k+kn}{import} \PYG{n+nn}{pyrockmodulus}\PYG{n+nn}{.}\PYG{n+nn}{rock\PYGZus{}variables} \PYG{k}{as} \PYG{n+nn}{ucs\PYGZus{}class}
\PYG{n}{ucs\PYGZus{}class}\PYG{o}{.}\PYG{n}{ucs\PYGZus{}strength\PYGZus{}criteria}\PYG{p}{(}\PYG{l+s+s1}{\PYGZsq{}}\PYG{l+s+s1}{ISRMCAT}\PYG{l+s+se}{\PYGZbs{}n}\PYG{l+s+s1}{1979}\PYG{l+s+s1}{\PYGZsq{}}\PYG{p}{)}
\end{sphinxVerbatim}

\sphinxAtStartPar
Output

\begin{sphinxVerbatim}[commandchars=\\\{\}]
\PYG{p}{(}\PYG{p}{[}\PYG{l+s+s1}{\PYGZsq{}}\PYG{l+s+s1}{R0}\PYG{l+s+s1}{\PYGZsq{}}\PYG{p}{,} \PYG{l+s+s1}{\PYGZsq{}}\PYG{l+s+s1}{R1}\PYG{l+s+s1}{\PYGZsq{}}\PYG{p}{,} \PYG{l+s+s1}{\PYGZsq{}}\PYG{l+s+s1}{R2}\PYG{l+s+s1}{\PYGZsq{}}\PYG{p}{,} \PYG{l+s+s1}{\PYGZsq{}}\PYG{l+s+s1}{R3}\PYG{l+s+s1}{\PYGZsq{}}\PYG{p}{,} \PYG{l+s+s1}{\PYGZsq{}}\PYG{l+s+s1}{R4}\PYG{l+s+s1}{\PYGZsq{}}\PYG{p}{,} \PYG{l+s+s1}{\PYGZsq{}}\PYG{l+s+s1}{R5}\PYG{l+s+s1}{\PYGZsq{}}\PYG{p}{,} \PYG{l+s+s1}{\PYGZsq{}}\PYG{l+s+s1}{R6}\PYG{l+s+s1}{\PYGZsq{}}\PYG{p}{]}\PYG{p}{,} \PYG{p}{[}\PYG{l+m+mf}{0.25}\PYG{p}{,} \PYG{l+m+mi}{1}\PYG{p}{,} \PYG{l+m+mi}{5}\PYG{p}{,} \PYG{l+m+mi}{25}\PYG{p}{,} \PYG{l+m+mi}{50}\PYG{p}{,} \PYG{l+m+mi}{100}\PYG{p}{,} \PYG{l+m+mi}{250}\PYG{p}{,} \PYG{l+m+mi}{1000}\PYG{p}{]}\PYG{p}{)}
\end{sphinxVerbatim}
\begin{enumerate}
\sphinxsetlistlabels{\arabic}{enumi}{enumii}{}{.}%
\setcounter{enumi}{1}
\item {} 
\sphinxAtStartPar
A horizontal bar like plot to show the various uniaxial strength classification systems.

\end{enumerate}

\begin{sphinxVerbatim}[commandchars=\\\{\}]
\PYG{k+kn}{import} \PYG{n+nn}{pyrockmodulus}\PYG{n+nn}{.}\PYG{n+nn}{ucs\PYGZus{}bar\PYGZus{}chart\PYGZus{}plot} \PYG{k}{as} \PYG{n+nn}{ucs\PYGZus{}classification\PYGZus{}plot}
\PYG{k+kn}{import} \PYG{n+nn}{matplotlib}\PYG{n+nn}{.}\PYG{n+nn}{pyplot} \PYG{k}{as} \PYG{n+nn}{plt}

\PYG{n}{ucs\PYGZus{}class} \PYG{o}{=} \PYG{n}{ucs\PYGZus{}classification\PYGZus{}plot}\PYG{o}{.}\PYG{n}{initial\PYGZus{}processing}\PYG{p}{(}\PYG{p}{)}
\PYG{n}{plt}\PYG{o}{.}\PYG{n}{show}\PYG{p}{(}\PYG{p}{)}
\end{sphinxVerbatim}

\sphinxAtStartPar
\sphinxincludegraphics{{example04}.png}

\sphinxstepscope


\chapter{pyrockmodulus}
\label{\detokenize{modules:pyrockmodulus}}\label{\detokenize{modules::doc}}
\sphinxstepscope


\section{pyrockmodulus package}
\label{\detokenize{pyrockmodulus:pyrockmodulus-package}}\label{\detokenize{pyrockmodulus::doc}}

\subsection{Submodules}
\label{\detokenize{pyrockmodulus:submodules}}

\subsubsection{Deere\sphinxhyphen{}Miller \sphinxhyphen{} Modulus Ratio (MR)}
\label{\detokenize{pyrockmodulus:module-pyrockmodulus.pyrockmodulus}}\label{\detokenize{pyrockmodulus:deere-miller-modulus-ratio-mr}}\index{module@\spxentry{module}!pyrockmodulus.pyrockmodulus@\spxentry{pyrockmodulus.pyrockmodulus}}\index{pyrockmodulus.pyrockmodulus@\spxentry{pyrockmodulus.pyrockmodulus}!module@\spxentry{module}}\index{modulus\_ratio (class in pyrockmodulus.pyrockmodulus)@\spxentry{modulus\_ratio}\spxextra{class in pyrockmodulus.pyrockmodulus}}

\begin{fulllineitems}
\phantomsection\label{\detokenize{pyrockmodulus:pyrockmodulus.pyrockmodulus.modulus_ratio}}
\pysigstartsignatures
\pysigline{\sphinxbfcode{\sphinxupquote{class\DUrole{w}{  }}}\sphinxcode{\sphinxupquote{pyrockmodulus.pyrockmodulus.}}\sphinxbfcode{\sphinxupquote{modulus\_ratio}}}
\pysigstopsignatures
\sphinxAtStartPar
Bases: \sphinxcode{\sphinxupquote{object}}

\sphinxAtStartPar
Based on the classification of Deere DU, Miller RP. Engineering Classification and Index Properties for Intact Rocks. Fort Belvoir, VA: Defense Technical Information Center; 1966.
Data digitization courtesy of Rohatgi, Ankit. “WebPlotDigitizer.” (2017).

\sphinxAtStartPar
\# ADVANCED: By assigning the \sphinxstyleemphasis{\_rocktype\_dictionary} variable, more control over the clusters being plotted is gained.
\index{abline() (pyrockmodulus.pyrockmodulus.modulus\_ratio method)@\spxentry{abline()}\spxextra{pyrockmodulus.pyrockmodulus.modulus\_ratio method}}

\begin{fulllineitems}
\phantomsection\label{\detokenize{pyrockmodulus:pyrockmodulus.pyrockmodulus.modulus_ratio.abline}}
\pysigstartsignatures
\pysiglinewithargsret{\sphinxbfcode{\sphinxupquote{abline}}}{\emph{\DUrole{n}{slope}}, \emph{\DUrole{n}{intercept}}, \emph{\DUrole{n}{dr\_state}}, \emph{\DUrole{n}{multiplier}\DUrole{o}{=}\DUrole{default_value}{1}}, \emph{\DUrole{n}{ratio}\DUrole{o}{=}\DUrole{default_value}{\textquotesingle{}\textquotesingle{}}}, \emph{\DUrole{n}{ax}\DUrole{o}{=}\DUrole{default_value}{None}}, \emph{\DUrole{n}{x\_text\_loc}\DUrole{o}{=}\DUrole{default_value}{0.15}}}{}
\pysigstopsignatures
\sphinxAtStartPar
Function to plot the slopped lines based on a slope and a y\sphinxhyphen{}intercept, basically mx+c. It is defined to form the Low/Avg/High MR ratio in the deere\sphinxhyphen{}miller classification plot.
\begin{quote}\begin{description}
\sphinxlineitem{Parameters}\begin{itemize}
\item {} 
\sphinxAtStartPar
\sphinxstyleliteralstrong{\sphinxupquote{slope}} (\sphinxstyleliteralemphasis{\sphinxupquote{float}}) \textendash{} the slope of the line

\item {} 
\sphinxAtStartPar
\sphinxstyleliteralstrong{\sphinxupquote{intercept}} (\sphinxstyleliteralemphasis{\sphinxupquote{float}}) \textendash{} the intercept of the lube

\item {} 
\sphinxAtStartPar
\sphinxstyleliteralstrong{\sphinxupquote{dr\_state}} (\sphinxstyleliteralemphasis{\sphinxupquote{str}}) \textendash{} draw state to move between the line drawing and the placement/writing of the text. Options {[}Line, Text{]}

\item {} 
\sphinxAtStartPar
\sphinxstyleliteralstrong{\sphinxupquote{multiplier}} (\sphinxstyleliteralemphasis{\sphinxupquote{int}}) \textendash{} in case of a need of a multiplier

\item {} 
\sphinxAtStartPar
\sphinxstyleliteralstrong{\sphinxupquote{ratio}} (\sphinxstyleliteralemphasis{\sphinxupquote{str}}) \textendash{} text associated with the MR modulus

\item {} 
\sphinxAtStartPar
\sphinxstyleliteralstrong{\sphinxupquote{ax}} (\sphinxstyleliteralemphasis{\sphinxupquote{matplotlib}}) \textendash{} Matplotlib Axis

\item {} 
\sphinxAtStartPar
\sphinxstyleliteralstrong{\sphinxupquote{x\_text\_loc}} (\sphinxstyleliteralemphasis{\sphinxupquote{float}}) \textendash{} slope to write text

\end{itemize}

\sphinxlineitem{Returns}
\sphinxAtStartPar


\sphinxlineitem{Return type}
\sphinxAtStartPar


\end{description}\end{quote}

\end{fulllineitems}

\index{deere\_miller\_clusters() (pyrockmodulus.pyrockmodulus.modulus\_ratio method)@\spxentry{deere\_miller\_clusters()}\spxextra{pyrockmodulus.pyrockmodulus.modulus\_ratio method}}

\begin{fulllineitems}
\phantomsection\label{\detokenize{pyrockmodulus:pyrockmodulus.pyrockmodulus.modulus_ratio.deere_miller_clusters}}
\pysigstartsignatures
\pysiglinewithargsret{\sphinxbfcode{\sphinxupquote{deere\_miller\_clusters}}}{\emph{\DUrole{n}{ax}}, \emph{\DUrole{n}{df\_of\_clusters\_deere\_miller}}, \emph{\DUrole{n}{r\_type}\DUrole{o}{=}\DUrole{default_value}{None}}, \emph{\DUrole{n}{plot\_all\_clusters\_bool}\DUrole{o}{=}\DUrole{default_value}{False}}}{}
\pysigstopsignatures
\sphinxAtStartPar
Load information needed to plot
\begin{quote}\begin{description}
\sphinxlineitem{Parameters}\begin{itemize}
\item {} 
\sphinxAtStartPar
\sphinxstyleliteralstrong{\sphinxupquote{ax}} (\sphinxstyleliteralemphasis{\sphinxupquote{matplotlib}}) \textendash{} Axis to plot on

\item {} 
\sphinxAtStartPar
\sphinxstyleliteralstrong{\sphinxupquote{df\_of\_clusters\_deere\_miller}} (\sphinxstyleliteralemphasis{\sphinxupquote{dict}}) \textendash{} will plot defined cluster. Options Sedimentary, Igneous, Metamorphic.

\item {} 
\sphinxAtStartPar
\sphinxstyleliteralstrong{\sphinxupquote{r\_type}} (\sphinxstyleliteralemphasis{\sphinxupquote{str}}) \textendash{} Define the rock type to be plotted. plot\_all\_clusters\_bool MUST be false.

\item {} 
\sphinxAtStartPar
\sphinxstyleliteralstrong{\sphinxupquote{plot\_all\_clusters\_bool}} (\sphinxstyleliteralemphasis{\sphinxupquote{bool}}) \textendash{} Plot all the clusters.

\end{itemize}

\sphinxlineitem{Returns}
\sphinxAtStartPar


\sphinxlineitem{Return type}
\sphinxAtStartPar


\end{description}\end{quote}

\end{fulllineitems}

\index{format\_axis() (pyrockmodulus.pyrockmodulus.modulus\_ratio method)@\spxentry{format\_axis()}\spxextra{pyrockmodulus.pyrockmodulus.modulus\_ratio method}}

\begin{fulllineitems}
\phantomsection\label{\detokenize{pyrockmodulus:pyrockmodulus.pyrockmodulus.modulus_ratio.format_axis}}
\pysigstartsignatures
\pysiglinewithargsret{\sphinxbfcode{\sphinxupquote{format\_axis}}}{\emph{\DUrole{n}{ax}}, \emph{\DUrole{n}{state}\DUrole{o}{=}\DUrole{default_value}{\textquotesingle{}\textquotesingle{}}}, \emph{\DUrole{n}{major\_axis\_vline}\DUrole{o}{=}\DUrole{default_value}{True}}}{}
\pysigstopsignatures
\sphinxAtStartPar
Format log\sphinxhyphen{}log Axis
\begin{quote}\begin{description}
\sphinxlineitem{Parameters}\begin{itemize}
\item {} 
\sphinxAtStartPar
\sphinxstyleliteralstrong{\sphinxupquote{ax}} (\sphinxstyleliteralemphasis{\sphinxupquote{matplotlib}}) \textendash{} Axis to plot on

\item {} 
\sphinxAtStartPar
\sphinxstyleliteralstrong{\sphinxupquote{state}} \textendash{} state to enable to disable slopped lines

\item {} 
\sphinxAtStartPar
\sphinxstyleliteralstrong{\sphinxupquote{major\_axis\_vline}} (\sphinxstyleliteralemphasis{\sphinxupquote{bool}}) \textendash{} Plot the major axis vlines

\end{itemize}

\sphinxlineitem{Returns}
\sphinxAtStartPar


\sphinxlineitem{Return type}
\sphinxAtStartPar


\end{description}\end{quote}

\end{fulllineitems}

\index{initial\_processing() (pyrockmodulus.pyrockmodulus.modulus\_ratio method)@\spxentry{initial\_processing()}\spxextra{pyrockmodulus.pyrockmodulus.modulus\_ratio method}}

\begin{fulllineitems}
\phantomsection\label{\detokenize{pyrockmodulus:pyrockmodulus.pyrockmodulus.modulus_ratio.initial_processing}}
\pysigstartsignatures
\pysiglinewithargsret{\sphinxbfcode{\sphinxupquote{initial\_processing}}}{\emph{\DUrole{n}{rock\_type\_to\_plot}\DUrole{o}{=}\DUrole{default_value}{None}}, \emph{\DUrole{n}{plot\_all\_clusters}\DUrole{o}{=}\DUrole{default_value}{False}}, \emph{\DUrole{n}{ucs\_class\_type}\DUrole{o}{=}\DUrole{default_value}{None}}, \emph{\DUrole{n}{ax}\DUrole{o}{=}\DUrole{default_value}{None}}}{}
\pysigstopsignatures\begin{quote}

\sphinxAtStartPar
Main function to plot the Modulus Ratio underlay
\begin{quote}\begin{description}
\sphinxlineitem{param rock\_type\_to\_plot}
\sphinxAtStartPar
Rock cluster type to plot.

\sphinxlineitem{type rock\_type\_to\_plot}
\sphinxAtStartPar
UCS Strength Criteria adopted. Options Sedimentary, Igneous, Metamorphic.

\sphinxlineitem{param ucs\_class\_type}
\sphinxAtStartPar
UCS Strength Criteria adopted. Options ‘ISRM

\end{description}\end{quote}
\end{quote}

\sphinxAtStartPar
1977’, ‘ISRMCAT
1979’, ‘Bieniawski
1974’, ‘Jennings
1973’, ‘Broch \& Franklin
1972’, ‘Geological Society
1970’, ‘Deere \& Miller
1966’, ‘Coates
1964’, ‘Coates \& Parsons
1966’, ‘ISO 14689
2017’, ‘Anon
1977’, ‘Anon
1979’, ‘Ramamurthy
2004’
\begin{quote}
\begin{quote}\begin{description}
\sphinxlineitem{type ucs\_class\_type}
\sphinxAtStartPar
str

\sphinxlineitem{param ax}
\sphinxAtStartPar
Axis to plot on

\sphinxlineitem{type ax}
\sphinxAtStartPar
matplotlib

\sphinxlineitem{return}
\sphinxAtStartPar
Axis

\sphinxlineitem{rtype}
\sphinxAtStartPar
Matplotlib Axis

\end{description}\end{quote}
\end{quote}

\end{fulllineitems}

\index{load\_data() (pyrockmodulus.pyrockmodulus.modulus\_ratio method)@\spxentry{load\_data()}\spxextra{pyrockmodulus.pyrockmodulus.modulus\_ratio method}}

\begin{fulllineitems}
\phantomsection\label{\detokenize{pyrockmodulus:pyrockmodulus.pyrockmodulus.modulus_ratio.load_data}}
\pysigstartsignatures
\pysiglinewithargsret{\sphinxbfcode{\sphinxupquote{load\_data}}}{\emph{\DUrole{n}{df\_deere\_miller\_data}}}{}
\pysigstopsignatures
\sphinxAtStartPar
Load the file that holds the digital deere\_miller cluster points. This information will be used to plot the deere\sphinxhyphen{}miller clusters based on the user requirements.
\begin{quote}\begin{description}
\sphinxlineitem{Parameters}
\sphinxAtStartPar
\sphinxstyleliteralstrong{\sphinxupquote{df\_deere\_miller\_data}} (\sphinxstyleliteralemphasis{\sphinxupquote{str}}) \textendash{} file path to the location of the csv

\sphinxlineitem{Returns}
\sphinxAtStartPar
dictionary containing the type of rock and the points that form its cluster.

\sphinxlineitem{Return type}
\sphinxAtStartPar
dict

\end{description}\end{quote}

\end{fulllineitems}

\index{plot\_clusters() (pyrockmodulus.pyrockmodulus.modulus\_ratio method)@\spxentry{plot\_clusters()}\spxextra{pyrockmodulus.pyrockmodulus.modulus\_ratio method}}

\begin{fulllineitems}
\phantomsection\label{\detokenize{pyrockmodulus:pyrockmodulus.pyrockmodulus.modulus_ratio.plot_clusters}}
\pysigstartsignatures
\pysiglinewithargsret{\sphinxbfcode{\sphinxupquote{plot\_clusters}}}{\emph{\DUrole{n}{k}}, \emph{\DUrole{n}{v}}, \emph{\DUrole{n}{ax}}, \emph{\DUrole{n}{df\_of\_clusters\_deere\_miller}}}{}
\pysigstopsignatures
\sphinxAtStartPar
Plot the clusters
\begin{quote}\begin{description}
\sphinxlineitem{Parameters}\begin{itemize}
\item {} 
\sphinxAtStartPar
\sphinxstyleliteralstrong{\sphinxupquote{k}} (\sphinxstyleliteralemphasis{\sphinxupquote{str}}) \textendash{} key

\item {} 
\sphinxAtStartPar
\sphinxstyleliteralstrong{\sphinxupquote{v}} (\sphinxstyleliteralemphasis{\sphinxupquote{str}}) \textendash{} value

\item {} 
\sphinxAtStartPar
\sphinxstyleliteralstrong{\sphinxupquote{ax}} (\sphinxstyleliteralemphasis{\sphinxupquote{matplotlib}}) \textendash{} Axis to plot on

\item {} 
\sphinxAtStartPar
\sphinxstyleliteralstrong{\sphinxupquote{df\_of\_clusters\_deere\_miller}} (\sphinxstyleliteralemphasis{\sphinxupquote{dict}}) \textendash{} dictionary containing the type of rock and the points that form its cluster.

\end{itemize}

\sphinxlineitem{Returns}
\sphinxAtStartPar


\sphinxlineitem{Return type}
\sphinxAtStartPar


\end{description}\end{quote}

\end{fulllineitems}

\index{plot\_v\_lines() (pyrockmodulus.pyrockmodulus.modulus\_ratio method)@\spxentry{plot\_v\_lines()}\spxextra{pyrockmodulus.pyrockmodulus.modulus\_ratio method}}

\begin{fulllineitems}
\phantomsection\label{\detokenize{pyrockmodulus:pyrockmodulus.pyrockmodulus.modulus_ratio.plot_v_lines}}
\pysigstartsignatures
\pysiglinewithargsret{\sphinxbfcode{\sphinxupquote{plot\_v\_lines}}}{\emph{\DUrole{n}{vlines}}, \emph{\DUrole{n}{ax}}}{}
\pysigstopsignatures
\sphinxAtStartPar
Plot lines and annotate the UCS Strength Criteria adopted
\begin{quote}\begin{description}
\sphinxlineitem{Parameters}\begin{itemize}
\item {} 
\sphinxAtStartPar
\sphinxstyleliteralstrong{\sphinxupquote{vlines}} (\sphinxstyleliteralemphasis{\sphinxupquote{list}}\sphinxstyleliteralemphasis{\sphinxupquote{{[}}}\sphinxstyleliteralemphasis{\sphinxupquote{float}}\sphinxstyleliteralemphasis{\sphinxupquote{{]}}}) \textendash{} Locations of V Lines

\item {} 
\sphinxAtStartPar
\sphinxstyleliteralstrong{\sphinxupquote{ax}} (\sphinxstyleliteralemphasis{\sphinxupquote{matplotlib}}) \textendash{} Axis to plot

\end{itemize}

\sphinxlineitem{Returns}
\sphinxAtStartPar


\sphinxlineitem{Return type}
\sphinxAtStartPar


\end{description}\end{quote}

\end{fulllineitems}


\end{fulllineitems}



\subsubsection{Tatone et al. \sphinxhyphen{} Strength Ratio (SR)}
\label{\detokenize{pyrockmodulus:module-0}}\label{\detokenize{pyrockmodulus:tatone-et-al-strength-ratio-sr}}\index{module@\spxentry{module}!pyrockmodulus.pyrockmodulus@\spxentry{pyrockmodulus.pyrockmodulus}}\index{pyrockmodulus.pyrockmodulus@\spxentry{pyrockmodulus.pyrockmodulus}!module@\spxentry{module}}\index{strength\_ratio (class in pyrockmodulus.pyrockmodulus)@\spxentry{strength\_ratio}\spxextra{class in pyrockmodulus.pyrockmodulus}}

\begin{fulllineitems}
\phantomsection\label{\detokenize{pyrockmodulus:pyrockmodulus.pyrockmodulus.strength_ratio}}
\pysigstartsignatures
\pysigline{\sphinxbfcode{\sphinxupquote{class\DUrole{w}{  }}}\sphinxcode{\sphinxupquote{pyrockmodulus.pyrockmodulus.}}\sphinxbfcode{\sphinxupquote{strength\_ratio}}}
\pysigstopsignatures
\sphinxAtStartPar
Bases: \sphinxcode{\sphinxupquote{object}}

\sphinxAtStartPar
Based on the classification of Tatone, B.S.A., Abdelaziz, A. \& Grasselli, G. Novel Mechanical Classification Method of Rock Based on the Uniaxial Compressive Strength and Brazilian Disc Strength. Rock Mech Rock Eng 55, 2503\textendash{}2507 (2022). \sphinxurl{https://doi.org/10.1007/s00603-021-02759-7}
Data was built using a bivariant KDE
\# \sphinxurl{https://docs.scipy.org/doc/scipy/reference/generated/scipy.stats.gaussian\_kde.html}
\# \sphinxurl{https://towardsdatascience.com/simple-example-of-2d-density-plots-in-python-83b83b934f67}

\sphinxAtStartPar
\# ADVANCED: By assigning the \sphinxstyleemphasis{\_rocktype\_dict} variable, more control over the clusters being plotted is gained.
\index{abline() (pyrockmodulus.pyrockmodulus.strength\_ratio method)@\spxentry{abline()}\spxextra{pyrockmodulus.pyrockmodulus.strength\_ratio method}}

\begin{fulllineitems}
\phantomsection\label{\detokenize{pyrockmodulus:pyrockmodulus.pyrockmodulus.strength_ratio.abline}}
\pysigstartsignatures
\pysiglinewithargsret{\sphinxbfcode{\sphinxupquote{abline}}}{\emph{\DUrole{n}{slope}}, \emph{\DUrole{n}{intercept}}, \emph{\DUrole{n}{dr\_state}}, \emph{\DUrole{n}{multiplier}\DUrole{o}{=}\DUrole{default_value}{1}}, \emph{\DUrole{n}{ratio}\DUrole{o}{=}\DUrole{default_value}{\textquotesingle{}\textquotesingle{}}}, \emph{\DUrole{n}{ax}\DUrole{o}{=}\DUrole{default_value}{None}}}{}
\pysigstopsignatures
\sphinxAtStartPar
Function to plot the slopped lines based on a slope and a y\sphinxhyphen{}intercept, basically mx+c. It is defined to form the Low/Avg/High MR ratio in the deere\sphinxhyphen{}miller classification plot.
\begin{quote}\begin{description}
\sphinxlineitem{Parameters}\begin{itemize}
\item {} 
\sphinxAtStartPar
\sphinxstyleliteralstrong{\sphinxupquote{slope}} (\sphinxstyleliteralemphasis{\sphinxupquote{float}}) \textendash{} the slope of the line

\item {} 
\sphinxAtStartPar
\sphinxstyleliteralstrong{\sphinxupquote{intercept}} (\sphinxstyleliteralemphasis{\sphinxupquote{float}}) \textendash{} the intercept of the lube

\item {} 
\sphinxAtStartPar
\sphinxstyleliteralstrong{\sphinxupquote{dr\_state}} (\sphinxstyleliteralemphasis{\sphinxupquote{str}}) \textendash{} draw state to move between the line drawing and the placement/writing of the text. Options {[}Line, Text{]}

\item {} 
\sphinxAtStartPar
\sphinxstyleliteralstrong{\sphinxupquote{multiplier}} (\sphinxstyleliteralemphasis{\sphinxupquote{int}}) \textendash{} in case of a need of a multiplier

\item {} 
\sphinxAtStartPar
\sphinxstyleliteralstrong{\sphinxupquote{ratio}} (\sphinxstyleliteralemphasis{\sphinxupquote{str}}) \textendash{} text associated with the MR modulus

\item {} 
\sphinxAtStartPar
\sphinxstyleliteralstrong{\sphinxupquote{ax}} (\sphinxstyleliteralemphasis{\sphinxupquote{matplotlib}}) \textendash{} Matplotlib Axis

\item {} 
\sphinxAtStartPar
\sphinxstyleliteralstrong{\sphinxupquote{x\_text\_loc}} (\sphinxstyleliteralemphasis{\sphinxupquote{float}}) \textendash{} slope to write text

\end{itemize}

\sphinxlineitem{Returns}
\sphinxAtStartPar


\sphinxlineitem{Return type}
\sphinxAtStartPar


\end{description}\end{quote}

\end{fulllineitems}

\index{format\_axis() (pyrockmodulus.pyrockmodulus.strength\_ratio method)@\spxentry{format\_axis()}\spxextra{pyrockmodulus.pyrockmodulus.strength\_ratio method}}

\begin{fulllineitems}
\phantomsection\label{\detokenize{pyrockmodulus:pyrockmodulus.pyrockmodulus.strength_ratio.format_axis}}
\pysigstartsignatures
\pysiglinewithargsret{\sphinxbfcode{\sphinxupquote{format\_axis}}}{\emph{\DUrole{n}{ax}}, \emph{\DUrole{n}{state}\DUrole{o}{=}\DUrole{default_value}{\textquotesingle{}\textquotesingle{}}}, \emph{\DUrole{n}{major\_axis\_vline}\DUrole{o}{=}\DUrole{default_value}{True}}}{}
\pysigstopsignatures
\sphinxAtStartPar
Format log\sphinxhyphen{}log Axis
\begin{quote}\begin{description}
\sphinxlineitem{Parameters}\begin{itemize}
\item {} 
\sphinxAtStartPar
\sphinxstyleliteralstrong{\sphinxupquote{ax}} (\sphinxstyleliteralemphasis{\sphinxupquote{matplotlib}}) \textendash{} Axis to plot on

\item {} 
\sphinxAtStartPar
\sphinxstyleliteralstrong{\sphinxupquote{state}} \textendash{} state to enable to disable slopped lines

\item {} 
\sphinxAtStartPar
\sphinxstyleliteralstrong{\sphinxupquote{major\_axis\_vline}} (\sphinxstyleliteralemphasis{\sphinxupquote{bool}}) \textendash{} Plot the major axis vlines

\end{itemize}

\sphinxlineitem{Returns}
\sphinxAtStartPar


\sphinxlineitem{Return type}
\sphinxAtStartPar


\end{description}\end{quote}

\end{fulllineitems}

\index{initial\_processing() (pyrockmodulus.pyrockmodulus.strength\_ratio method)@\spxentry{initial\_processing()}\spxextra{pyrockmodulus.pyrockmodulus.strength\_ratio method}}

\begin{fulllineitems}
\phantomsection\label{\detokenize{pyrockmodulus:pyrockmodulus.pyrockmodulus.strength_ratio.initial_processing}}
\pysigstartsignatures
\pysiglinewithargsret{\sphinxbfcode{\sphinxupquote{initial\_processing}}}{\emph{\DUrole{n}{rock\_type\_to\_plot}\DUrole{o}{=}\DUrole{default_value}{None}}, \emph{\DUrole{n}{plot\_all\_clusters}\DUrole{o}{=}\DUrole{default_value}{False}}, \emph{\DUrole{n}{ucs\_class\_type}\DUrole{o}{=}\DUrole{default_value}{None}}, \emph{\DUrole{n}{ax}\DUrole{o}{=}\DUrole{default_value}{None}}}{}
\pysigstopsignatures\begin{quote}

\sphinxAtStartPar
Main function to plot the Modulus Ratio underlay
\begin{quote}\begin{description}
\sphinxlineitem{param rock\_type\_to\_plot}
\sphinxAtStartPar
Rock cluster type to plot.

\sphinxlineitem{type rock\_type\_to\_plot}
\sphinxAtStartPar
UCS Strength Criteria adopted. Options Sedimentary, Igneous, Metamorphic.

\sphinxlineitem{param ucs\_class\_type}
\sphinxAtStartPar
UCS Strength Criteria adopted. Options ‘ISRM

\end{description}\end{quote}
\end{quote}

\sphinxAtStartPar
1977’, ‘ISRMCAT
1979’, ‘Bieniawski
1974’, ‘Jennings
1973’, ‘Broch \& Franklin
1972’, ‘Geological Society
1970’, ‘Deere \& Miller
1966’, ‘Coates
1964’, ‘Coates \& Parsons
1966’, ‘ISO 14689
2017’, ‘Anon
1977’, ‘Anon
1979’, ‘Ramamurthy
2004’
\begin{quote}
\begin{quote}\begin{description}
\sphinxlineitem{type ucs\_class\_type}
\sphinxAtStartPar
str

\sphinxlineitem{param ax}
\sphinxAtStartPar
Axis to plot on

\sphinxlineitem{type ax}
\sphinxAtStartPar
matplotlib

\sphinxlineitem{return}
\sphinxAtStartPar
Axis

\sphinxlineitem{rtype}
\sphinxAtStartPar
Matplotlib Axis

\end{description}\end{quote}
\end{quote}

\end{fulllineitems}


\end{fulllineitems}



\subsubsection{Poisson’s Ratio and Density Plots}
\label{\detokenize{pyrockmodulus:module-1}}\label{\detokenize{pyrockmodulus:poisson-s-ratio-and-density-plots}}\index{module@\spxentry{module}!pyrockmodulus.pyrockmodulus@\spxentry{pyrockmodulus.pyrockmodulus}}\index{pyrockmodulus.pyrockmodulus@\spxentry{pyrockmodulus.pyrockmodulus}!module@\spxentry{module}}\index{poisson\_density (class in pyrockmodulus.pyrockmodulus)@\spxentry{poisson\_density}\spxextra{class in pyrockmodulus.pyrockmodulus}}

\begin{fulllineitems}
\phantomsection\label{\detokenize{pyrockmodulus:pyrockmodulus.pyrockmodulus.poisson_density}}
\pysigstartsignatures
\pysigline{\sphinxbfcode{\sphinxupquote{class\DUrole{w}{  }}}\sphinxcode{\sphinxupquote{pyrockmodulus.pyrockmodulus.}}\sphinxbfcode{\sphinxupquote{poisson\_density}}}
\pysigstopsignatures
\sphinxAtStartPar
Bases: \sphinxcode{\sphinxupquote{object}}

\sphinxAtStartPar
Load Poisson Ratio and Density information
\index{initial\_processing() (pyrockmodulus.pyrockmodulus.poisson\_density method)@\spxentry{initial\_processing()}\spxextra{pyrockmodulus.pyrockmodulus.poisson\_density method}}

\begin{fulllineitems}
\phantomsection\label{\detokenize{pyrockmodulus:pyrockmodulus.pyrockmodulus.poisson_density.initial_processing}}
\pysigstartsignatures
\pysiglinewithargsret{\sphinxbfcode{\sphinxupquote{initial\_processing}}}{}{}
\pysigstopsignatures
\sphinxAtStartPar
Load the variables and initialise the dataframe.
\begin{quote}\begin{description}
\sphinxlineitem{Returns}
\sphinxAtStartPar
DataFrame containing the Min/Max Poisson Ratio and the Min/Max Density divided by Rock Name nad ROck Group. The latter two impact the y\sphinxhyphen{}axis and the hbars and titles.

\sphinxlineitem{Return type}
\sphinxAtStartPar
pandas.DataFrame

\end{description}\end{quote}

\end{fulllineitems}

\index{plot\_span\_chart() (pyrockmodulus.pyrockmodulus.poisson\_density method)@\spxentry{plot\_span\_chart()}\spxextra{pyrockmodulus.pyrockmodulus.poisson\_density method}}

\begin{fulllineitems}
\phantomsection\label{\detokenize{pyrockmodulus:pyrockmodulus.pyrockmodulus.poisson_density.plot_span_chart}}
\pysigstartsignatures
\pysiglinewithargsret{\sphinxbfcode{\sphinxupquote{plot\_span\_chart}}}{\emph{\DUrole{n}{df\_to\_plot}}, \emph{\DUrole{n}{variable\_span}}, \emph{\DUrole{n}{variable\_label}}, \emph{\DUrole{n}{variable\_units}}, \emph{\DUrole{n}{ax}\DUrole{o}{=}\DUrole{default_value}{None}}}{}
\pysigstopsignatures
\sphinxAtStartPar
Plot a chart divided by the rock type and rock group.
\begin{quote}\begin{description}
\sphinxlineitem{Parameters}\begin{itemize}
\item {} 
\sphinxAtStartPar
\sphinxstyleliteralstrong{\sphinxupquote{df\_to\_plot}} (\sphinxstyleliteralemphasis{\sphinxupquote{pandas.DataFrame}}) \textendash{} Panda Dataframe to plot

\item {} 
\sphinxAtStartPar
\sphinxstyleliteralstrong{\sphinxupquote{variable\_span}} (\sphinxstyleliteralemphasis{\sphinxupquote{list}}\sphinxstyleliteralemphasis{\sphinxupquote{{[}}}\sphinxstyleliteralemphasis{\sphinxupquote{str}}\sphinxstyleliteralemphasis{\sphinxupquote{, }}\sphinxstyleliteralemphasis{\sphinxupquote{str}}\sphinxstyleliteralemphasis{\sphinxupquote{{]}}}) \textendash{} Span (i.e., min and max values) passed as a list. Must be the Column Header name in the DataFrame!

\item {} 
\sphinxAtStartPar
\sphinxstyleliteralstrong{\sphinxupquote{variable\_label}} (\sphinxstyleliteralemphasis{\sphinxupquote{str}}) \textendash{} Variable Name. X axis label

\item {} 
\sphinxAtStartPar
\sphinxstyleliteralstrong{\sphinxupquote{variable\_units}} (\sphinxstyleliteralemphasis{\sphinxupquote{str}}) \textendash{} Variable Units. X axis label unit

\item {} 
\sphinxAtStartPar
\sphinxstyleliteralstrong{\sphinxupquote{ax}} (\sphinxstyleliteralemphasis{\sphinxupquote{Matplolib}}) \textendash{} Matplotlib Axis to plot On

\end{itemize}

\sphinxlineitem{Returns}
\sphinxAtStartPar
Matplotlib AxesSubplots

\sphinxlineitem{Return type}
\sphinxAtStartPar
Matplotlib Axis

\end{description}\end{quote}

\end{fulllineitems}


\end{fulllineitems}



\subsubsection{UCS Classification Systems}
\label{\detokenize{pyrockmodulus:module-pyrockmodulus.rock_variables}}\label{\detokenize{pyrockmodulus:ucs-classification-systems}}\index{module@\spxentry{module}!pyrockmodulus.rock\_variables@\spxentry{pyrockmodulus.rock\_variables}}\index{pyrockmodulus.rock\_variables@\spxentry{pyrockmodulus.rock\_variables}!module@\spxentry{module}}\index{ucs\_strength\_criteria() (in module pyrockmodulus.rock\_variables)@\spxentry{ucs\_strength\_criteria()}\spxextra{in module pyrockmodulus.rock\_variables}}

\begin{fulllineitems}
\phantomsection\label{\detokenize{pyrockmodulus:pyrockmodulus.rock_variables.ucs_strength_criteria}}
\pysigstartsignatures
\pysiglinewithargsret{\sphinxcode{\sphinxupquote{pyrockmodulus.rock\_variables.}}\sphinxbfcode{\sphinxupquote{ucs\_strength\_criteria}}}{\emph{\DUrole{n}{type}}}{}
\pysigstopsignatures
\sphinxAtStartPar
\#\# Insert all UCS Strength Criterion Here.     \#\# ALL VALUES ARE IN MPa
\# Name Format \{Reference Name: {[}Name of Category{]}\}
\# Value Format \{Reference Name: {[}Boundaries Location{]}\}  \textless{}=\textgreater{} in MPa
\# converted\_psi {[}Reference name that are converted from psi to MPa{]}
\begin{quote}\begin{description}
\sphinxlineitem{Parameters}
\sphinxAtStartPar
\sphinxstyleliteralstrong{\sphinxupquote{type}} (\sphinxstyleliteralemphasis{\sphinxupquote{str}}) \textendash{} rock classification system to load

\sphinxlineitem{Returns}
\sphinxAtStartPar


\sphinxlineitem{Return type}
\sphinxAtStartPar


\end{description}\end{quote}

\end{fulllineitems}



\subsection{Supporting Modules}
\label{\detokenize{pyrockmodulus:supporting-modules}}

\subsubsection{pyrockmodulus.formatting\_codes}
\label{\detokenize{pyrockmodulus:module-pyrockmodulus.formatting_codes}}\label{\detokenize{pyrockmodulus:pyrockmodulus-formatting-codes}}\index{module@\spxentry{module}!pyrockmodulus.formatting\_codes@\spxentry{pyrockmodulus.formatting\_codes}}\index{pyrockmodulus.formatting\_codes@\spxentry{pyrockmodulus.formatting\_codes}!module@\spxentry{module}}\index{bold\_text() (in module pyrockmodulus.formatting\_codes)@\spxentry{bold\_text()}\spxextra{in module pyrockmodulus.formatting\_codes}}

\begin{fulllineitems}
\phantomsection\label{\detokenize{pyrockmodulus:pyrockmodulus.formatting_codes.bold_text}}
\pysigstartsignatures
\pysiglinewithargsret{\sphinxcode{\sphinxupquote{pyrockmodulus.formatting\_codes.}}\sphinxbfcode{\sphinxupquote{bold\_text}}}{\emph{\DUrole{n}{val}}}{}
\pysigstopsignatures
\sphinxAtStartPar
Returns text as bold
\begin{quote}\begin{description}
\sphinxlineitem{Parameters}
\sphinxAtStartPar
\sphinxstyleliteralstrong{\sphinxupquote{val}} (\sphinxstyleliteralemphasis{\sphinxupquote{str}}) \textendash{} Text

\sphinxlineitem{Returns}
\sphinxAtStartPar
Text as bold

\sphinxlineitem{Return type}
\sphinxAtStartPar
str

\end{description}\end{quote}

\end{fulllineitems}

\index{calc\_timer\_values() (in module pyrockmodulus.formatting\_codes)@\spxentry{calc\_timer\_values()}\spxextra{in module pyrockmodulus.formatting\_codes}}

\begin{fulllineitems}
\phantomsection\label{\detokenize{pyrockmodulus:pyrockmodulus.formatting_codes.calc_timer_values}}
\pysigstartsignatures
\pysiglinewithargsret{\sphinxcode{\sphinxupquote{pyrockmodulus.formatting\_codes.}}\sphinxbfcode{\sphinxupquote{calc\_timer\_values}}}{\emph{\DUrole{n}{end\_time}}}{}
\pysigstopsignatures
\sphinxAtStartPar
Function to calculate the time
\begin{quote}\begin{description}
\sphinxlineitem{Parameters}
\sphinxAtStartPar
\sphinxstyleliteralstrong{\sphinxupquote{end\_time}} (\sphinxstyleliteralemphasis{\sphinxupquote{float}}) \textendash{} Time (Difference in time in seconds)

\sphinxlineitem{Returns}
\sphinxAtStartPar
Time in minutes and seconds

\sphinxlineitem{Return type}
\sphinxAtStartPar
float

\end{description}\end{quote}

\end{fulllineitems}

\index{docstring\_creator() (in module pyrockmodulus.formatting\_codes)@\spxentry{docstring\_creator()}\spxextra{in module pyrockmodulus.formatting\_codes}}

\begin{fulllineitems}
\phantomsection\label{\detokenize{pyrockmodulus:pyrockmodulus.formatting_codes.docstring_creator}}
\pysigstartsignatures
\pysiglinewithargsret{\sphinxcode{\sphinxupquote{pyrockmodulus.formatting\_codes.}}\sphinxbfcode{\sphinxupquote{docstring\_creator}}}{\emph{\DUrole{n}{df}}}{}
\pysigstopsignatures
\sphinxAtStartPar
Write the example output for a docstring DataFrame
\begin{quote}\begin{description}
\sphinxlineitem{Parameters}
\sphinxAtStartPar
\sphinxstyleliteralstrong{\sphinxupquote{df}} (\sphinxstyleliteralemphasis{\sphinxupquote{pandas.DataFrame}}) \textendash{} DataFrame to be read

\sphinxlineitem{Returns}
\sphinxAtStartPar
prints the docstring and type for each element in the DataFrame

\sphinxlineitem{Return type}
\sphinxAtStartPar
str

\end{description}\end{quote}

\end{fulllineitems}

\index{green\_text() (in module pyrockmodulus.formatting\_codes)@\spxentry{green\_text()}\spxextra{in module pyrockmodulus.formatting\_codes}}

\begin{fulllineitems}
\phantomsection\label{\detokenize{pyrockmodulus:pyrockmodulus.formatting_codes.green_text}}
\pysigstartsignatures
\pysiglinewithargsret{\sphinxcode{\sphinxupquote{pyrockmodulus.formatting\_codes.}}\sphinxbfcode{\sphinxupquote{green\_text}}}{\emph{\DUrole{n}{val}}}{}
\pysigstopsignatures
\sphinxAtStartPar
Returns text as bold in green font color
\begin{quote}\begin{description}
\sphinxlineitem{Parameters}
\sphinxAtStartPar
\sphinxstyleliteralstrong{\sphinxupquote{val}} (\sphinxstyleliteralemphasis{\sphinxupquote{str}}) \textendash{} Text

\sphinxlineitem{Returns}
\sphinxAtStartPar
Text as bold in green font color

\sphinxlineitem{Return type}
\sphinxAtStartPar
str

\end{description}\end{quote}

\end{fulllineitems}

\index{print\_progress() (in module pyrockmodulus.formatting\_codes)@\spxentry{print\_progress()}\spxextra{in module pyrockmodulus.formatting\_codes}}

\begin{fulllineitems}
\phantomsection\label{\detokenize{pyrockmodulus:pyrockmodulus.formatting_codes.print_progress}}
\pysigstartsignatures
\pysiglinewithargsret{\sphinxcode{\sphinxupquote{pyrockmodulus.formatting\_codes.}}\sphinxbfcode{\sphinxupquote{print\_progress}}}{\emph{\DUrole{n}{iteration}}, \emph{\DUrole{n}{total}}, \emph{\DUrole{n}{prefix}\DUrole{o}{=}\DUrole{default_value}{\textquotesingle{}\textquotesingle{}}}, \emph{\DUrole{n}{suffix}\DUrole{o}{=}\DUrole{default_value}{\textquotesingle{}\textquotesingle{}}}, \emph{\DUrole{n}{decimals}\DUrole{o}{=}\DUrole{default_value}{1}}, \emph{\DUrole{n}{bar\_length}\DUrole{o}{=}\DUrole{default_value}{50}}}{}
\pysigstopsignatures
\sphinxAtStartPar
Call in a loop to create terminal progress bar
Adjusted bar length to 50, to display on small screen
\begin{quote}\begin{description}
\sphinxlineitem{Parameters}\begin{itemize}
\item {} 
\sphinxAtStartPar
\sphinxstyleliteralstrong{\sphinxupquote{iteration}} (\sphinxstyleliteralemphasis{\sphinxupquote{int}}) \textendash{} current iteration

\item {} 
\sphinxAtStartPar
\sphinxstyleliteralstrong{\sphinxupquote{total}} (\sphinxstyleliteralemphasis{\sphinxupquote{int}}) \textendash{} total iteration

\item {} 
\sphinxAtStartPar
\sphinxstyleliteralstrong{\sphinxupquote{prefix}} (\sphinxstyleliteralemphasis{\sphinxupquote{str}}) \textendash{} prefix string

\item {} 
\sphinxAtStartPar
\sphinxstyleliteralstrong{\sphinxupquote{suffix}} (\sphinxstyleliteralemphasis{\sphinxupquote{str}}) \textendash{} suffix string

\item {} 
\sphinxAtStartPar
\sphinxstyleliteralstrong{\sphinxupquote{decimals}} (\sphinxstyleliteralemphasis{\sphinxupquote{int}}) \textendash{} positive number of decimals in percent complete

\item {} 
\sphinxAtStartPar
\sphinxstyleliteralstrong{\sphinxupquote{bar\_length}} (\sphinxstyleliteralemphasis{\sphinxupquote{int}}) \textendash{} character length of bar

\end{itemize}

\sphinxlineitem{Returns}
\sphinxAtStartPar
system output showing progress

\sphinxlineitem{Return type}
\sphinxAtStartPar


\end{description}\end{quote}

\end{fulllineitems}

\index{red\_text() (in module pyrockmodulus.formatting\_codes)@\spxentry{red\_text()}\spxextra{in module pyrockmodulus.formatting\_codes}}

\begin{fulllineitems}
\phantomsection\label{\detokenize{pyrockmodulus:pyrockmodulus.formatting_codes.red_text}}
\pysigstartsignatures
\pysiglinewithargsret{\sphinxcode{\sphinxupquote{pyrockmodulus.formatting\_codes.}}\sphinxbfcode{\sphinxupquote{red\_text}}}{\emph{\DUrole{n}{val}}}{}
\pysigstopsignatures
\sphinxAtStartPar
Returns text as bold in red font color
\begin{quote}\begin{description}
\sphinxlineitem{Parameters}
\sphinxAtStartPar
\sphinxstyleliteralstrong{\sphinxupquote{val}} (\sphinxstyleliteralemphasis{\sphinxupquote{str}}) \textendash{} Text

\sphinxlineitem{Returns}
\sphinxAtStartPar
Text as bold in red font color

\sphinxlineitem{Return type}
\sphinxAtStartPar
str

\end{description}\end{quote}

\end{fulllineitems}



\subsubsection{pyrockmodulus.ucs\_bar\_chart\_plot}
\label{\detokenize{pyrockmodulus:module-pyrockmodulus.ucs_bar_chart_plot}}\label{\detokenize{pyrockmodulus:pyrockmodulus-ucs-bar-chart-plot}}\index{module@\spxentry{module}!pyrockmodulus.ucs\_bar\_chart\_plot@\spxentry{pyrockmodulus.ucs\_bar\_chart\_plot}}\index{pyrockmodulus.ucs\_bar\_chart\_plot@\spxentry{pyrockmodulus.ucs\_bar\_chart\_plot}!module@\spxentry{module}}\index{initial\_processing() (in module pyrockmodulus.ucs\_bar\_chart\_plot)@\spxentry{initial\_processing()}\spxextra{in module pyrockmodulus.ucs\_bar\_chart\_plot}}

\begin{fulllineitems}
\phantomsection\label{\detokenize{pyrockmodulus:pyrockmodulus.ucs_bar_chart_plot.initial_processing}}
\pysigstartsignatures
\pysiglinewithargsret{\sphinxcode{\sphinxupquote{pyrockmodulus.ucs\_bar\_chart\_plot.}}\sphinxbfcode{\sphinxupquote{initial\_processing}}}{}{}
\pysigstopsignatures
\end{fulllineitems}

\index{my\_path (in module pyrockmodulus.ucs\_bar\_chart\_plot)@\spxentry{my\_path}\spxextra{in module pyrockmodulus.ucs\_bar\_chart\_plot}}

\begin{fulllineitems}
\phantomsection\label{\detokenize{pyrockmodulus:pyrockmodulus.ucs_bar_chart_plot.my_path}}
\pysigstartsignatures
\pysigline{\sphinxcode{\sphinxupquote{pyrockmodulus.ucs\_bar\_chart\_plot.}}\sphinxbfcode{\sphinxupquote{my\_path}}\sphinxbfcode{\sphinxupquote{\DUrole{w}{  }\DUrole{p}{=}\DUrole{w}{  }\textquotesingle{}/hdd/home/aly/Desktop/Dropbox/Python\_Codes/digital\_modulus\_strength\_ratio/pyrockmodulus\textquotesingle{}}}}
\pysigstopsignatures
\sphinxAtStartPar
Default MATPLOTLIB Fonts

\end{fulllineitems}



\chapter{Indices and tables}
\label{\detokenize{index:indices-and-tables}}\begin{itemize}
\item {} 
\sphinxAtStartPar
\DUrole{xref,std,std-ref}{genindex}

\item {} 
\sphinxAtStartPar
\DUrole{xref,std,std-ref}{modindex}

\item {} 
\sphinxAtStartPar
\DUrole{xref,std,std-ref}{search}

\end{itemize}


\renewcommand{\indexname}{Python Module Index}
\begin{sphinxtheindex}
\let\bigletter\sphinxstyleindexlettergroup
\bigletter{p}
\item\relax\sphinxstyleindexentry{pyrockmodulus.formatting\_codes}\sphinxstyleindexpageref{pyrockmodulus:\detokenize{module-pyrockmodulus.formatting_codes}}
\item\relax\sphinxstyleindexentry{pyrockmodulus.pyrockmodulus}\sphinxstyleindexpageref{pyrockmodulus:\detokenize{module-1}}
\item\relax\sphinxstyleindexentry{pyrockmodulus.rock\_variables}\sphinxstyleindexpageref{pyrockmodulus:\detokenize{module-pyrockmodulus.rock_variables}}
\item\relax\sphinxstyleindexentry{pyrockmodulus.ucs\_bar\_chart\_plot}\sphinxstyleindexpageref{pyrockmodulus:\detokenize{module-pyrockmodulus.ucs_bar_chart_plot}}
\end{sphinxtheindex}

\renewcommand{\indexname}{Index}
\printindex
\end{document}